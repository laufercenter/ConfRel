
\begin{thebibliography}{99}

\bibitem{Elber2007}
Elber, R. A Milestoning Study of the Kinetics of an Allosteric Transition: Atomically Detailed Simulations of Deoxy Scapharca 
Hemoglobin. Biophysical J., 2007, 92, 85-87.

\bibitem{Moult2011}
Moult, J.; Fidelis, K.; Kryshtafovych, A.; Tramontano, A. Critical assessment of methods of protein structure prediction (CASP)-round IX.
Proteins, 2011, 79, 1-5.

\bibitem{West2007}
West, A.M.; Elber, R.; Shalloway, D. Extending molecular dynamics time scales with milestoning: example of complex kinetics 
in a solvated peptide. J Chem Phys. 2007, 126, 145104-1451014.

\bibitem{Haas2007}
Haas, K.; Chu, J.W. Decomposition of energy and free energy changes by following the flow of work along reaction path.
J. Chem. Phys. 2009, 131, 144105-144111.

\bibitem{Jonsson1998}
Jónsson, H.; Mills, G.; Jacobsen, K.W. Nudged Elastic Band Method for Finding Minimum Energy Paths of Transitions, 
in Classical and Quantum Dynamics in Condensed Phase Simulations, Ed. B. J. Berne, G. Ciccotti and D. F. 
Coker, 385 (World Scientific, 1998).

\bibitem{E2007}
E, W.; Ren, W.; Vanden-Eijnden, E. Simplified and improved string method for computing the minimum energy paths in 
barrier-crossing eventsi. J. Chem. Phys. 2007, 126, 164103.

\bibitem{Dellago2002}
Dellago, C.; Bolhuis, P.G.; Geissler, P.L. Transition Path Sampling, Adv. Chem. Phys. 2002, 123, 1-84. 

\bibitem{Cheng2006}
Cheng, X.; Wang, H.; Grant, B.; Sine, S.M.; McCammon, J.A. Targeted Molecular Dynamics Study of C-Loop Closure 
and Channel Gating in Nicotinic Receptors. 2006, 9, 134. 

\bibitem{Elber2005}
Elber, R. Long-timescale simulation methods. Cur. Opi. in Str. Biol. 2005, 15, 151-156.

\bibitem{Torrie1977}
Torrie, G. M.; Valleau, J. P. Nonphysical sampling distributions in Monte Carlo free-energy estimation: Umbrella sampling
(1977) J. Comput. Phys. 23, 187

\bibitem{Tironi1994}
Tironi, I.G.; van Gunsteren, W.F. A molecular-dynamics simulation study of chloroform. Mol. Phys. 1994, 83, 381-403.

\bibitem{Meirovitch2007}
Meirovitch, H. Recent developments in methodologies for calculating the entropy and free energy of biological systems by computer simulation.
Current Opinion in Structural Biology, 2007, 17, 181-186.


\bibitem{Mobley2007}
Mobley, D.L.; Chodera, J.D.; Dill, K.A. The combining and release method: obtaining correct binding free energies in 
the presence of protein conformational change. Journal of Chemical Theory and Computation 2007, 3, 1231-1235.

\bibitem{Mobley2012}
Mobley, D.L.; Klimovich, P.V. Perspective: Alchemical free energy calculations for drug discovery. 
J. Chem. Phys. 2012, 137, 230901-12.

\bibitem{Mobley2006}
Mobley, D.L.; Chodera, J.D.; Dill, K.A. On the use of orientational restraints and symmetry corrections in alchemical 
free energy calculations. J. Chem. Phy. 2006, 125, 084902. 

\bibitem{Chipot2007}
Chipot, C.; Shell, M.S.; Pohorille, A. Introduction, in Chipot, C., Pohorille, A., editors. Free Energy
Calculations: Theory and Applications in Chemistry and Biology. Springer Series in Chemical
Physics, vol. 86. Berlin and Heidelberg: Springer; 2007, p. 1–32.

\bibitem{Jorgensen2004}
Jorgensen, W.L. The many roles of computation in drug discovery, Science 2004, 303, 1813–8.

\bibitem{Gilson2007}
Gilson, M.K.; Zhou, H.X. Calculation of protein-ligand binding affinities. Annu Rev Biophys Biomol Struct. (2007) 36, 21-42.

\bibitem{Dill1997}
Dill, K.A.; H.S. Chan.  From Levinthal to Pathways to Funnels:  The "New View" of Protein Folding Kinetics.  Nature Structural Biology 4, 10-19 (1997)

\bibitem{Dill2008}
Dill, K.A.; Ozkan, S.B.; Shell, M.S.; Weikl, T.R. The protein folding problem. Annual Review of Biophysics (2008), 37, 289-316.

\bibitem{Anfinsen1973}
Anfinsen. C.B. Principles that Govern the Folding of Protein Chains. Science (1973) 181, 223-230.

\bibitem{Christ2007}
Christ, C.D.; van Gunsteren, W.F. Enveloping distribution sampling: A method to calculate free energy differences from a single simulation,
J. Chem. Phys. (2007), 126, 184110.

\bibitem{Ytreberg2006}
Ytreberg, F.; Zuckerman, D. Simple estimation of absolute free energies for biomolecules. J. Chem. Phys. 2006, 124, 104105.

\bibitem{Park2008}
Park, S.; Lau, A.; Roux, B. Computing conformational free energy by deactivated morphing. J. Chem. Phys. 2008, 129, 134102

\bibitem{Zheng2008}
Zheng, L.; Chen, M.; Yang, W. Random walk in orthogonal space to achieve efficient free-energy simulation of complex systems, Proc. Natl. Acad. Sci. 2008, 105 (51), 20227.

\bibitem{Christ2007}
Christ, C.D.; van Gunsteren, W.F. Enveloping distribution sampling: A method to calculate free energy differences from a 
single simulation. 2007, J. Chem. Phys. 126, 184110.

\bibitem{Shell2010}
Shell, S.M. A replica-exchange approach to computing peptide conformational free energies. Mol. Sim. 2010, 7, 505-515.

\bibitem{Tyka2006}
Tyka, M.; Clarke, A.; Sessions, R. An Efficient, Path-Independent Method for Free-Energy Calculations. J.Phys.Chem. B 2006, 110, 17212-17220.

\bibitem{Cecchini2009}
Cecchini, M., Krivov, S.V., Spichty, M., Karplus, M. Calculation of free-energy differences by confinement simulations. Application to peptide conformers. J. Phys. Chem. B. 2009, 113, 9728-9740.

\bibitem{Ovchinnikov2013}
Ovchinnikov, V.; Cecchini, M.; Karplus, M. A Simplified Confinement Method for Calculating Absolute Free Energies 
and Free Energy and Entropy Differences. J. Phys. Chem. B. 2013, 117, 750-762.

\bibitem{Spichty2010}
Spichty, M.; Cecchini, M.; Karplus, M. Conformational Free-Energy Difference of a Miniprotein from Nonequilibrium 
Simulations. J. Phys. Chem. Lett., 2010, 1, 1922-1926.

\bibitem{Strajbl2000}
Strajbl, M.; Sham, Y.Y.; Villà, J.; Chu, Z.-T.; Warshel, A. Calculations of Activation Entropies of Chemical Reactions 
in Solution. (2000) 104, 4578-4584.  


\bibitem{Krivov2004} 
Krivov, S.; Karplus, M. Hidden complexity of free energy surfaces for peptide (protein) folding Proc. Natl. Acad. Sci. U.S.A. 2004, 101, (41), 14766.

\bibitem{Alexander2007}
Alexander, P.A.; He, Y.; Chen, Y.; Orban, J. Bryan, P. The design and characterization of two proteins with $88 \%$ sequence identity but different 
structure and function. Proc. Natl. Acad. Sci. 2007, 104 (29), 11963-11968.

\bibitem{He2008}
He, Y.; Chen, Y.; Alexander, P.A.; Orban, J. NMR structures of two designed proteins with high sequence identity but different fold and function. Proc. Natl. Acad. Sci. 2008, 105 (38), 14412-14417.

\bibitem{Alexander2009}
Alexander, P.A.; He, Y.; Chen, Y.; Orban, J. Bryan, P. A minimal sequence code for switching protein structure and function. Proc. Natl. Acad. Sci. 2009, 106(50), 21149-21154.

\bibitem{Bryan2010}
Bryan, P.N.; Orban, J. Proteins that switch folds. Curr Opin Struct Biol. 2010, 20(4), 482-488.

\bibitem{He2012}
He, Y.; Chen, Y.; Alexander, P.A.; Bryan, P.N.; Orban, J. Mutational tipping points for switching protein folds and functions. Structure. 2012, 20(2), 283-91.

\bibitem{Shortle2009}
Shortle, D. One sequence plus one mutation equals two folds. Proc. Natl. Acad. Sci. 2009, 106(50), 21011-21012. 

\bibitem{Sheffler2009}
Sheffler, W.; Baker, D. RosettaHoles: Rapid assessment of protein core packing for structure prediction, refinement, design, and validation. Protein Science. 2009, 18(1), 229-239.

\bibitem{Case2012}
D.A. Case, T.A. Darden, T.E. Cheatham, III, C.L. Simmerling, J. Wang, R.E. Duke, R. Luo, R.C. Walker, W. Zhang, K.M. Merz, B. Roberts, S. Hayik, A. Roitberg, G. Seabra, J. Swails, A.W. Goetz, I. Kolossváry, K.F. Wong, F. Paesani, J. Vanicek, R.M. Wolf, J. Liu, X. Wu, S.R. Brozell, T. Steinbrecher, H. Gohlke, Q. Cai, X. Ye, J. Wang, M.-J. Hsieh, G. Cui, D.R. Roe, D.H. Mathews, M.G. Seetin, R. Salomon-Ferrer, C. Sagui, V. Babin, T. Luchko, S. Gusarov, A. Kovalenko, and P.A. Kollman (2012), AMBER 12, University of California, San Francisco.

\bibitem{Goetz2012}
Goetz, A.W.; Williamson, M.J.; Xu, D.; Poole, D.; Le Grand, S.; Walker, R.C. Routine microsecond molecular dynamics simulations with AMBER - Part I: Generalized 
Born. J. Chem. Theory Comput. 2012, 8(5) 1542.

\bibitem{MacCallum2011}
MacCallum, J.; Perez, A.; Schnieders, MJ.; Hua, L.; Jacobson, M.P.; Dill, K.A. Assessment of protein structure refinement 
in CASP9. Proteins, 2011, 79, 74-90.

\bibitem{Kryshtafovych2011}
Kryshtafovych, A.; Fidelis, K; and Tramontano, A. Evaluation of model quality predictions in CASP9. Proteins, 2011, 79, 91–106

\bibitem{Zemla2003}
Zemla, A. LGA: a method for finding 3D similarities in protein structures. Nucleic Acids Res 2003, 31, 3370–3374.

\bibitem{Perez2012}
Perez, A.; Yang, Z.; Bahar, I.; Dill, K.A.; MacCallum, J.L.; FlexE: Using Elastic Network Models to Compare Models of Protein Structure. J. Chem. Theory Comput., 2012, 8, 3985-3991. 

\bibitem{Wang2011}
Wang, Q.; Vantasin, K.; Xu, D.; Shang, Y. MUFOLD-WQA: A new selective consensus method for quality assessment in protein structure prediction. 
Proteins, 2011, 79: 185–195. 

\bibitem{Case2005}
Case, D.A.; Cheatham, III, T.E.; Darden, T.; Gohlke, Luo, H.R.; Merz, Jr., K.M.;  Onufriev, A; Simmerling, C.; 
Wang, B.; R. Woods, R. The Amber biomolecular simulation programs. J. Computat. Chem. (20005) 26, 1668-1688.



\end{thebibliography}

\end{document}

