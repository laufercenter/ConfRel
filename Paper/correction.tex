\bibitem{Handl2009}
Handl, J.; Knowles, J.; Lovel, S.C. Artefacts and biases affecting the evaluation of scoring 
functions on decoy sets for protein structure prediction. Bioinformatics, 2009, 25, 1271-1279.

\bibitem{Hornak2006}
Hornak, V.; Abel, R.; Okur, A.; Strockbine, B.; Roitberg, A.; Simmerling, C. Comparison of multiple Amber force fields 
and development of improved protein backbone parameters. Proteins., 2006, 65, 712-725.

\bibitem{Mongan2006}
Mongan, J.; Simmerling, C.; A. McCammon, J.; A. Case, D.; Onufriev, A. Generalized
Born with a simple, robust molecular volume correction. J. Chem. Theory Comput.,
2006, 3, 156-169.

\section{Method}

The confinement method has been described in details in ref. by Tyka et al. \cite{Tyka2006} and
Cecchini et  al. \cite{Cecchini2009}. The basic approach of the confinement approach is the same in
both these papers. A thermodynamic cycle is used to compute the free energy between conformations A 
and B. However, there are some small technical differences between the two approaches. Here we briefly 
describe the procedure that we used. 


\begin{enumerate}

\item  In the first step, a minimization of A and B conformations are performed. These minimized conformations (A* and B*)
       are the reference conformation of that state. 

\item  The free energy of confining the ensemble (A or B) to a microstate (A* or B*) is
       calculated. This is done by gradually applying larger and larger
       harmonic restraints on all the atoms of the biomolecule. This is done by running 21 molecular dynamics simulation
       (each 20 ns long) for each leg of the thermodynamic cycle, where the harmonic restraint force constant was scaled from 0.00005
       Kcal/Mol (mostly free) to 81.92 Kcal/Mol (frozen in one microstate). In this final restrained state, the
       rotational contribution to the free energy is frozen out
       and the only remaining contribution is the vibrational part. The free energy for this step is
       estimated from the fluctuations around
       the reference structure using a numerical
       approach developed by Tyka et. al. \cite{Tyka2006}. The confinement free energy calculated in
       this way is recorded as
       $\Delta G_{A,A*}$ and $\Delta G_{B,B*}$ as shown in Figure~\ref{fig:method}.

\item  Finally the thermodynamic cycle is closed by calculating the free energy between the final
       restrained state A* and B* using normal mode analysis or quasiharmonic analysis. The free energy calculated in
       this way is shown as $\Delta G_{A*,B*}$ in Figure~\ref{fig:method}.

\item  The full free energy, $\Delta G_{A,B}$ between the two state A and B is calculated using the equation
       $\Delta G_{A,B}$ = $\Delta G_{A,A*}$ - $\Delta G_{B,B*}$ + $\Delta G_{A*,B*}$

\end{enumerate}

All calculations where performed with the amber 11 suit of programs \cite{Case2012},\cite{Goetz2012} in 
combination with ff99SB forcefield\cite{Hornak2006} and generalized born implicit solvent \cite{Mongan2006}.
Interestingly, we extend the method for calculation of per residue free
energy in an approximate way. For this purpose, the confinement energy, $\Delta G_{A,A*}$ and
$\Delta G_{B,B*}$ of each residue is calculated in the usual numerical way as described in ref. by Tyka 
et al. \cite{Tyka2006}. The internal energy of each residue is calculated using the decomp module of amber from the final
restrained trajectory. We call this method approximate as we ignore the entropic contribution from the normal
mode or quasiharmonic analysis. However, this contrbution to the total free energy is much smaller,
which allow us to study the mechanistic details of conformational preference of each residue.

