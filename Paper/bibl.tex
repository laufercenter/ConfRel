
\begin{thebibliography}{99}

\bibitem{Meirovitch2007}
Meirovitch, H. Recent developments in methodologies for calculating the entropy and free energy of biological systems by computer simulation.
Current Opinion in Structural Biology, 2007, 17, 181-186.

\bibitem{Chipot2007}
Chipot, C.; Shell, M.S.; Pohorille, A. Introduction, in Chipot, C., Pohorille, A., editors. Free Energy
Calculations: Theory and Applications in Chemistry and Biology. Springer Series in Chemical
Physics, vol. 86. Berlin and Heidelberg: Springer; 2007, p. 1–32.

\bibitem{Jorgensen2004}
Jorgensen, W.L. The many roles of computation in drug discovery, Science 2004, 303, 1813–8.

\bibitem{Gilson2007}
Gilson, M.K.; Zhou, H.X. Calculation of protein-ligand binding affinities. Annu Rev Biophys Biomol Struct. (2007) 36, 21-42.

\bibitem{Torrie1977}
Torrie, G. M.; Valleau, J. P. iNonphysical sampling distributions in Monte Carlo free-energy estimation: Umbrella sampling 
(1977) J. Comput. Phys. 23, 187

\bibitem{Tironi1994}
Tironi, I.G.; van Gunsteren, W.F. A molecular-dynamics simulation study of chloroform. Mol. Phys. (1994) 83, 381-403.

\bibitem{Dill1997}
Dill, K.A.; H.S. Chan.  From Levinthal to Pathways to Funnels:  The "New View" of Protein Folding Kinetics.  Nature Structural Biology 4, 10-19 (1997)

\bibitem{Dill2008}
Dill, K.A.; Ozkan, S.B.; Shell, M.S.; Weikl, T.R. The protein folding problem. Annual Review of Biophysics (2008), 37, 289-316.

\bibitem{Anfinsen1973}
Anfinsen. C.B. Principles that Govern the Folding of Protein Chains. Science (1973) 181, 223-230.

\bibitem{Christ2007}
Christ, C.D.; van Gunsteren, W.F. Enveloping distribution sampling: A method to calculate free energy differences from a single simulation,
J. Chem. Phys. (2007), 126, 184110.

\bibitem{Ytreberg2006}
Ytreberg, F.; Zuckerman, D. Simple estimation of absolute free energies for biomolecules. J. Chem. Phys. 2006, 124, 104105.

\bibitem{Park2008}
Park, S.; Lau, A.; Roux, B. Computing conformational free energy by deactivated morphing. J. Chem. Phys. 2008, 129, 134102

\bibitem{Zheng2008}
Zheng, L.; Chen, M.; Yang, W. Random walk in orthogonal space to achieve efficient free-energy simulation of complex systems, Proc. Natl. Acad. Sci. 2008, 105 (51), 20227.

\bibitem{Tyka2006}
Tyka, M.; Clarke, A.; Sessions, R. An Efficient, Path-Independent Method for Free-Energy Calculations. J.Phys.Chem. B 2006, 110, 17212-17220.

\bibitem{Cecchini2009}
Cecchini, M., Krivov, S.V., Spichty, M., Karplus, M. Calculation of free-energy differences by confinement simulations. Application to peptide conformers. 
J. Phys. Chem. B 113, p. 9728-9740 (2009).

\bibitem{Strajbl2000}
Strajbl, M.; Sham, Y.Y.; Villà, J.; Chu, Z.-T.; Warshel, A. Calculations of Activation Entropies of Chemical Reactions 
in Solution. (2000) 104, 4578-4584.  

\bibitem{Krivov2004} 
Krivov, S.; Karplus, M. Hidden complexity of free energy surfaces for peptide (protein) folding Proc. Natl. Acad. Sci. U.S.A. 2004, 101, (41), 14766.

\bibitem{Alexander2007}
Alexander, P.A.; He, Y.; Chen, Y.; Orban, J. Bryan, P. The design and characterization of two proteins with $88 \%$ sequence identity but different 
structure and function. Proc. Natl. Acad. Sci. 2007, 104 (29), 11963-11968.

\bibitem{He2008}
He, Y.; Chen, Y.; Alexander, P.A.; Orban, J. NMR structures of two designed proteins with high sequence identity but different fold and function. Proc. Natl. Acad. Sci. 2008, 105 (38), 14412-14417.

\bibitem{Alexander2009}
Alexander, P.A.; He, Y.; Chen, Y.; Orban, J. Bryan, P. A minimal sequence code for switching protein structure and function. Proc. Natl. Acad. Sci. 2009, 106(50), 21149-21154.

\bibitem{Shortle20009}
Shortle, D. One sequence plus one mutation equals two folds. Proc. Natl. Acad. Sci. 2009, 106(50), 21011-21012. 

\bibitem{Sheffler2009}
Sheffler, W.; Baker, D. RosettaHoles: Rapid assessment of protein core packing for structure prediction, refinement, design, and validation. Protein Science. 2009, 18(1), 229-239.


\bibitem{MacCallum2011}
MacCallum, J.; Pérez, A.; Schnieders, MJ.; Hua, L.; Jacobson, M.P.; Dill, K.A. Assessment of protein structure refinement 
in CASP9. Proteins, 2011, 79, 74-90.

\bibitem{Kryshtafovych2011}
Kryshtafovych, A.; Fidelis, K; and Tramontano, A. Evaluation of model quality predictions in CASP9. Proteins, 2011, 79, 91–106

\bibitem{Zemla2003}
Zemla, A. LGA: a method for finding 3D similarities in protein structures. Nucleic Acids Res 2003, 31, 3370–3374.


\bibitem{Perez2012}
Perez, A.; Yang, Z.; Bahar, I.; Dill, K.A.; MacCallum, J.L.; FlexE: Using Elastic Network Models to Compare Models of Protein Structure. J. Chem. Theory Comput., 2012, 8, 3985-3991. 

\bibitem{Wang2011}
Wang, Q.; Vantasin, K.; Xu, D.; Shang, Y. MUFOLD-WQA: A new selective consensus method for quality assessment in protein structure prediction. 
Proteins, 2011, 79: 185–195. 

\bibitem{Case2005}
Case, D.A.; Cheatham, III, T.E.; Darden, T.; Gohlke, Luo, H.R.; Merz, Jr., K.M.;  Onufriev, A; Simmerling, C.; 
Wang, B.; R. Woods, R. The Amber biomolecular simulation programs. J. Computat. Chem. (20005) 26, 1668-1688.



\end{thebibliography}

\end{document}

